\documentclass{supervision}
\usepackage{course}

\Supervision{1}

\begin{document}
  \begin{questions}
    \section*{Definitions}
    \question In just a few words, define each term and specify what it is
      useful for.
      \begin{parts}
        \part idempotent
        \part location transparency
        \part marshalling
        \part pure, impure names
        \part batching
        \part middleware
        \part clock skew
        \part clock drift
      \end{parts}

    \section*{NFS}
    \question
      \begin{parts}
        \part Write a 300-word description of what NFS is, how it works, and
          when you might use it and why. Please read around the subject a bit.
          No cut-n-paste from other resources!
        \part What were the most important changes from NFS v2 to v3? Why were
          these changes significant, and what were their effects?
      \end{parts}

    \section*{Time}
    \question For each of the following uses of time, what does the time need
      to be measured relative to (e.g. UT1, UTC, and NTP server on a LAN, a
      local oscillator)?
      \begin{parts}
        \part local process scheduling
        \part local I/O
        \part network protcols
        \part cryptographic certificate/ticket generation
        \part performance profiling
      \end{parts}

    \section*{NTP}
    \question Write a Java program which acts as an NTP client and prints the
      current time, as estimated from the NTP server, to the console whenever
      it is run. For example:

      \begin{code}{sh}
        bash $ java -jar current-time.jar
        Fri Feb 24 15:24:54 GMT 2012
        bash $
      \end{code}

      You will need to send UDP packets in an appropriate format to an NTP
      server. The Computer Lab has a set of NTP servers which you may wish to
      use:

      \begin{code}{}
        server ntp0.cl.cam.ac.uk
        server ntp1a.cl.cam.ac.uk
        server ntp1b.cl.cam.ac.uk
        server ntp1c.cl.cam.ac.uk
        server ntp1d.cl.cam.ac.uk
      \end{code}

      Please make sure your packets conform to NTP version 3 or later (see RFC
      1305 or later) and that you don't send more than two packets to the
      server each time you run your program.

    \section*{Vector clocks}
    \question
      \begin{parts}
        \part Given the sequence of messages below, show the value of the
          Lamport and vector clocks at each node at each send or receive event
          it participates in.

          % TABLE HERE
        \part Using the Lamport and vector clocks calculated above, state
          whether or not the following events can be determined to have a
          \emph{happens-before} relationship.
          % GRAPH HERE
  \end{questions}
\end{document}
