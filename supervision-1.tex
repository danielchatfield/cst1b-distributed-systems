\documentclass{supervision}
\usepackage{course}

\Supervision{1}

\begin{document}
  \begin{questions}
    \section*{Definitions}
    \question In just a few words, define each term and specify what it is
      useful for.
      \begin{parts}
        \part idempotent
          \begin{solution}
            An operation is idempotent if applying it multiple times results in
            the same change as doing it once.
          \end{solution}

        \part location transparency
          \begin{solution}
            Location transparency is using a logical name to identify a network
            resource that is independant of physical location.
          \end{solution}

        \part marshalling
          \begin{solution}
            Marshalling is the process of transforming an object in memory into
            a format suitable for storage or transmission.

            It is useful for sending arguments over RMI.
          \end{solution}

        \part pure, impure names
          \begin{solution}
            A pure name contains no information such as the location of the
            named object. In contrast, an impure name does contain such
            information. An example of a pure name would be a UID such as
            \lstinline|wngdc7mwad8128a9nxpctbc7j0tmvs6w|. An example of an
            impure name would be \lstinline|dc584@cam.ac.uk|.
          \end{solution}

        \part batching
          \begin{solution}
            Batching is where operations are delayed so that multiple
            operations can be sent together.

            By batching operations overheads are minimised.
          \end{solution}

        \part middleware
          \begin{solution}
            Middleware is a layer between the OS and distributed applications.
            It hides the complexity of the system and bridges the gap between
            low-level OS communications and programming langauage abstractions.
          \end{solution}

        \part clock skew
          \begin{solution}
            Clock skew is the difference between the times on two clocks.
          \end{solution}

        \part clock drift
          \begin{solution}
            Clock drift is the rate at which a clock drifts from perfect time.
          \end{solution}
      \end{parts}

    \section*{NFS}
    \question
      \begin{parts}
        \part Write a 300-word description of what NFS is, how it works, and
          when you might use it and why. Please read around the subject a bit.
          No cut-n-paste from other resources!
          \begin{solution}
            NFS is a distributed file system developed at Sun. It allows users
            to access files on a remote host in the same way they would access
            local files by providing an abstraction that makes the file access
            completely transparent to the client.

            When someone accesses a file over NFS the kernerl will make an RPC
            call to the server machine, marshalling the arguments (user,
            file handle etc.).

            NFS (v2 and v3) was stateless which meant that faul recovery was
            easy. If the server crashed then the client would simply have to
            wait for it to reboot and send the RPC again. If a client crashes,
            the server does not need to do anything at all - no cleanup is
            required.

            By deciding to make the protocol stateless, no operation could have
            implicit arguments (e.g. a read call had to have the filehandler).
            This had the side effect of making those operations idempotent -
            rather than requesting the next $x$ bytes, the client had to
            explicity pass the offset. Idempotency means that the server
            natively tolerated network packet duplication and RPC retries for
            those operations.

            There are many reasons why you would want to use NFS, but they all
            boil down to wanting to have some data on one machine be accessible
            from other remote machines. This is usually to reduce duplication,
            ensure consistency or facilitating network backups.
          \end{solution}

        \part What were the most important changes from NFS v2 to v3? Why were
          these changes significant, and what were their effects?
          \begin{solution}
            Version 3 added support for

            \begin{itemize}
              \item 64-bit file sizes and thus files larger than 2GB could be
                read.
              \item asynchronous writes (improved performance)
              \item a READDIRPLUS operation that returned additional data while
                scanning a directory to minimize the need for further calls.
              \item TCP was added as a transport which made using NFS over a
                WAN more feasible
            \end{itemize}
          \end{solution}
      \end{parts}

    \section*{Time}
    \question For each of the following uses of time, what does the time need
      to be measured relative to (e.g. UT1, UTC, an NTP server on a LAN, a
      local oscillator)?
      \begin{parts}
        \part local process scheduling
          \begin{solution}
            A local oscillator
          \end{solution}

        \part local I/O
          \begin{solution}
            A local oscillator
          \end{solution}

        \part network protcols
          \begin{solution}
            UT1
          \end{solution}
        \part cryptographic certificate/ticket generation
          \begin{solution}
            UT1
          \end{solution}
        \part performance profiling
          \begin{solution}
            A local oscillator
          \end{solution}
      \end{parts}

    \section*{NTP}
    \question Write a Java program which acts as an NTP client and prints the
      current time, as estimated from the NTP server, to the console whenever
      it is run. For example:

      \begin{code}{sh}
        bash $ java -jar current-time.jar
        Fri Feb 24 15:24:54 GMT 2012
        bash $
      \end{code}

      You will need to send UDP packets in an appropriate format to an NTP
      server. The Computer Lab has a set of NTP servers which you may wish to
      use:

      \begin{code}{}
        server ntp0.cl.cam.ac.uk
        server ntp1a.cl.cam.ac.uk
        server ntp1b.cl.cam.ac.uk
        server ntp1c.cl.cam.ac.uk
        server ntp1d.cl.cam.ac.uk
      \end{code}

      Please make sure your packets conform to NTP version 3 or later (see RFC
      1305 or later) and that you don't send more than two packets to the
      server each time you run your program.

    \section*{Vector clocks}
    \question
      \begin{parts}
        \part Given the sequence of messages below, show the value of the
          Lamport and vector clocks at each node at each send or receive event
          it participates in.

          \begin{solution}
            \begin{code}{}
              ___________________________________________________
              |      |  LAMPORT  |             VECTOR            |
              |event | A| B| C| D|   A   |   B   |   C   |   D   |
              |------+--+--+--+--+-------+-------+-------+-------|
              |start | 0| 0| 0| 0|0,0,0,0|0,0,0,0|0,0,0,0|0,0,0,0|
              |------+--+--+--+--+-------+-------+-------+-------|
              |A -> B| 1| 2| -| -|1,0,0,0|1,1,0,0|   -   |   -   |
              |A -> C| 2| -| 3| -|2,0,0,0|   -   |2,0,1,0|   -   |
              |A -> D| 3| -| -| 4|3,0,0,0|   -   |   -   |3,0,0,1|
              |B -> D| -| 3| -| 5|   -   |1,2,0,0|   -   |3,2,0,2|
              |B -> C| -| 4| 5| -|   -   |1,3,0,0|2,3,2,0|   -   |
              |C -> D| -| -| 6| 7|   -   |   -   |2,3,3,0|3,3,3,3|
              |C -> A| 8| -| 7| -|4,3,4,0|   -   |2,3,4,0|   -   |
              |A -> B| 9|10| -| -|5,3,4,0|5,4,4,0|   -   |   -   |
              |C -> B| -|11| 8| -|   -   |5,5,5,0|2,3,5,0|   -   |
              |------+--+--+--+--+-------+-------+-------+-------|
              |finish| 9|11| 8| 7|5,3,4,0|5,5,5,0|2,3,5,0|3,3,3,3|
              |------+--+--+--+--+-------+-------+-------+-------|
            \end{code}
          \end{solution}
        \part Using the Lamport and vector clocks calculated above, state
          whether or not the following events can be determined to have a
          \emph{happens-before} relationship.
          \begin{solution}
            \begin{enumerate}
              \part | A -> C | A -> D |   no
              \part | A -> D | B -> C |   no  
              \part | B -> C | C -> B |  yes 
            \end{enumerate}
          \end{solution}
      \end{parts}
  \end{questions}
\end{document}
